\documentclass[12pt]{article}
\usepackage[utf8]{inputenc}
\usepackage{amsmath, amssymb}
\usepackage{graphicx}
\usepackage{hyperref}
\usepackage{geometry}
\usepackage{algorithm}
\usepackage{algpseudocode}
\usepackage{amsmath}
\usepackage{mhchem}
\geometry{margin=1in}

\title{Hartree-Fock and MP2 Implementation Report}
\author{Arnav Brahmasandra, Jay Shen, Enoch Woldu}
\date{\today}

\begin{document}

\maketitle

\begin{abstract}
This report documents the implementation of the Hartree-Fock (HF) and Møller-Plesset perturbation theory (MP2) methods as developed in this repository. The focus is on the theoretical background, algorithmic details, and computational results obtained from the code.
\end{abstract}

\section{Introduction}
The Hartree-Fock method is a foundational approach in quantum chemistry for approximating the electronic structure of atoms and molecules. MP2 provides a post-Hartree-Fock correction to account for electron correlation effects. This report describes the implementation and results of these methods.

\section{Theoretical Background}
\subsection{Hartree-Fock Method}
The Hartree-Fock (HF) method approximates the many-electron wavefunction as a single Slater determinant of molecular orbitals, leading to a set of coupled integro-differential equations known as the HF equations. These equations are solved iteratively using the self-consistent field (SCF) procedure: starting from an initial guess for the orbitals, the Fock matrix is constructed, diagonalized to obtain new orbitals, and the process is repeated until convergence. Basis sets, typically composed of atomic orbitals or Gaussian functions, are used to represent the molecular orbitals and make the calculations tractable.

\subsection{MP2 Correction}
The Møller-Plesset perturbation theory of second order (MP2) provides a systematic way to include electron correlation effects that are neglected in the Hartree-Fock approximation. In MP2, the total electronic energy is corrected by adding a second-order perturbative term, which accounts for the interactions between electron pairs that are not captured by the mean-field HF approach. The MP2 energy correction is computed using the HF molecular orbitals and their corresponding energies, and involves summing over all possible double excitations from occupied to virtual orbitals. This correction typically lowers the total energy and yields more accurate results for molecular properties, especially in systems where electron correlation plays a significant role.

\section{Implementation Details}

\subsection{Code Structure}
Outline the organization of the codebase and key modules.

\subsection{Algorithms}

\begin{algorithm}
\caption{Hartree-Fock Self-Consistent Field (SCF) Method}
\begin{algorithmic}[1]

\Statex \textbf{Input:} Basis set $\{\phi\}$; Coordinates $\{\vec{r}\}$
\Statex \textbf{Output:} Molecular orbitals; Energy

\State $P_{ij} \gets $ Guess
\State $S_{ij} \gets \langle \phi_i | \phi_j \rangle$ \Comment{PySCF}
\State $G_{ijkl} \gets (\phi_i\phi_j|\phi_k\phi_l)$ \Comment{PySCF}

\While{not converged}
    \State $F_{ij} \gets \langle \phi_i \: | \: T + \sum_k^{\text{nuclei}}V_k \: |\: \phi_j \rangle + \sum_{kl} P_{kl} \Big[ G_{ijkl} - \frac{1}{2} G_{ikjl} \Big]$

\EndWhile


\end{algorithmic}
\end{algorithm}

\section{Results}
Using our code, we calculated the energies of different molecular systems, where we calculated the energy and compared these results with the established literature values
\begin{itemize}
    \item Energy of $\ce{H2O}$: -84.2310577768
    \item Energy of $\ce{H2}$:
\end{itemize}
Present sample results, such as total energies for test molecules, and compare HF and MP2 results.

\section{Conclusion}
Summarize findings and discuss possible improvements or future work.

\section*{References}
List references to textbooks, articles, or documentation used.

\end{document}