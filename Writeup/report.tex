\documentclass[12pt]{article}
\usepackage[utf8]{inputenc}
\usepackage{amsmath, amssymb}
\usepackage{graphicx}
\usepackage{hyperref}
\usepackage{geometry}
\geometry{margin=1in}

\title{Hartree-Fock and MP2 Implementation Report}
\author{Arnav Brahmasandra, Jay Shen, Enoch Woldu}
\date{\today}

\begin{document}

\maketitle

\begin{abstract}
This report documents the implementation of the Hartree-Fock (HF) and Møller-Plesset perturbation theory (MP2) methods as developed in this repository. The focus is on the theoretical background, algorithmic details, and computational results obtained from the code.
\end{abstract}

\section{Introduction}
The Hartree-Fock method is a foundational approach in quantum chemistry for approximating the electronic structure of atoms and molecules. MP2 provides a post-Hartree-Fock correction to account for electron correlation effects. This report describes the implementation and results of these methods.

\section{Theoretical Background}
\subsection{Hartree-Fock Method}
Briefly describe the HF equations, self-consistent field (SCF) procedure, and the role of basis sets.

\subsection{MP2 Correction}
Summarize the MP2 energy correction and its significance in improving upon HF results.

\section{Implementation Details}

\subsection{Code Structure}
Outline the organization of the codebase and key modules.

\subsection{Algorithms}
Describe the main algorithms used for SCF convergence and MP2 energy calculation.

\section{Results}
Present sample results, such as total energies for test molecules, and compare HF and MP2 results.

\section{Conclusion}
Summarize findings and discuss possible improvements or future work.

\section*{References}
List references to textbooks, articles, or documentation used.

\end{document}